%------------------------------------------------------------------------------
\documentclass[master]{cimt}
% オプションについては,マニュアルを参照.
% \documentclass[master,oneside]{cimt} 

% 必要とするパッケージがあれば,ここで指定する.
\usepackage{graphicx}
\usepackage{url}

\usepackage{layout}

% 論文タイトル
\jtitle{Fuga: ランサムウェアからリアクティブに \\ データを保護するシステムの設計と実装}
% 長い時には自動的に改行されるが,次のように明示的に改行することもできる.
% \jtitle{長いタイトルを\\このように改行位置を指定して組む}

% 英文タイトル
\etitle{Fuga: Design and Implementation of \\ Reactive Data Protection System against Ransomware}

% 学籍番号
\studentid{48-236427}

% 著者名
\jauthor{手塚 尚哉}

% 英文著者名
\eauthor{Naoya Tezuka}

% 指導教員
\supervisor{落合秀也 准教授}

% 提出月..
\handin{2025}{1}


\begin{document}

% 表紙と表紙裏
\maketitle

% ここから前文
\frontmatter

% 概要
\begin{jabstract}
  \input doc/0-1.jabst.tex
\end{jabstract}

% 英文概要
\begin{eabstract}
  \input doc/0-2.eabst.tex
\end{eabstract}

% 目次
\tableofcontents
\listoffigures
\listoftables

% ここから本文
\mainmatter

\input doc/1.intro.tex
\input doc/2.related.tex
\input doc/3.ransomware_defense.tex
\input doc/4.approach.tex
\input doc/5.design.tex
\input doc/6.impl.tex

% ここから後付
\backmatter

% 発表文献
\pubUseLongName % 指定すると,タイトルが 「発表文献と研究活動」になる.
\begin{publications}
  \input doc/7-1.publications.tex
\end{publications}

% 参考文献: BibTeX を使う場合の例 (styleは適宜選択)
\bibliographystyle{junsrt}
\bibliography{doc/sampleJ}

% 参考文献: 直接記述する場合の例
% \input biblio.tex

% 謝辞 (前文においても良い)
\begin{acknowledgements}
  \input doc/7-2.ack.tex
\end{acknowledgements}

%付録 (必要な場合のみ)
\appendix

\input doc/7-3-1.appendix_eBPF.tex

\end{document}
