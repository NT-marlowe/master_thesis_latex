\chapter{結論}
\section{まとめ}
本研究では,今日ますます増大するランサムウェアの脅威に対する防御を提供するために,
ランサムウェアによる侵害からデータを保護する手法を提案した.
本手法ではランサムウェア攻撃の見逃しを減らす方針で調整された検知手法を利用してランサムウェアの疑いがあるユーザプロセスを検出し,
それらの挙動に対してリアクティブに,ファイル単位でデータのバックアップを作成して
ランサムウェアから隔離された領域に退避させる.
これにより従来のスナップショット方式の課題であったデータ損失と運用コストを解決した.
本研究ではeBPFという技術を活用し,提案手法をLinux上で動作するアプリケーションとして実装した.
実装の際には予備実験において明らかになったボトルネックの高速化を行った.
そして,ランサムウェアの機能を模した簡易なプログラムを用いて提案手法のデータ保護性能と性能上のオーバーヘッドを評価し,
提案手法の実装がランサムウェアの暗号化スピードに対して十分なデータ退避性能をもつことと,
提案手法を動作させるホストマシンに与える性能上の影響は軽微であることを確認した.

\section{今後の課題}
本研究では,提案手法の設計に対し,Linux環境を想定して実装を行なった.
しかしランサムウェア攻撃の対象の7割以上がWindows OSを標的としている \cite{trendmicro-report} ことを考慮すると,
実用性の観点からWindows環境における提案手法の実装を検討する必要がある.
本稿で示したLinux環境での実装はLinuxカーネルの実装に強く依存しており,
Windows環境への移植には大幅な改修が必要である.

提案手法の設計では,ランサムウェアによる侵害から隔離されたデータ領域であるData Shelterを想定した.
本稿で示した実装において,Data Shelterは管理者権限を持つユーザのみがアクセス可能なディレクトリとして実装したが,
これはData Shelterの実装として唯一の方法ではない.
例えば初回の書き込みより後の書き込みを許可しないファイルシステムの実装や,ローカルストレージではなく
リモートホストにネットワーク経由で退避データを送信するといった方法が考えられる.
特にリモートホストを利用する方法では,ランサムウェアに感染したホストと退避先のホストを物理的に隔離することができるため
Data Shelterの信頼性を向上させることができるが,ホスト間の通信のスループットや遅延がデータ保護性能に影響を与える可能性があり,
その影響を評価する必要がある.

提案手法が保護する対象のファイルの条件を設定可能にすることで手法の実用性を向上させることができると期待される.
設定の単位としてはファイルパスベースのフィルタリング,つまり特定のパスまたはディレクトリ以下に存在するファイルのみを保護対象とする方法が考えられる.
このような設定が可能になると,スナップショットによってファイルシステムやOS全体の状態を保存しておき,
頻繁に更新される高価値なファイルを提案手法によって保護するという補完的な運用がより効率的になる.
また,提案手法によって完全に保護することが難しい大きなファイルを保護対象から除外することで,
そのようなファイルを退避することによるその他のファイルの保護性能への影響を軽減することができる.
