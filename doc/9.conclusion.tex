\chapter{結論}
\label{chap:conclusion}
\section{まとめ}
本研究では,今日ますます増大するランサムウェアの脅威に対する防御を提供するために,
ランサムウェアによる侵害からデータを保護する手法としてFugaを提案した.
Fugaではランサムウェア攻撃の見逃しを減らす方針で調整された検知手法を利用してランサムウェアの疑いがあるユーザプロセスを特定する.
そして,Fugaはそのプロセスの挙動に対して,ファイル単位でリアクティブにデータのバックアップを作成し
ランサムウェアから隔離された領域に退避させる.
このアプローチにより,従来のスナップショット方式の課題であった運用コストとデータ損失を解決する.
Fugaの実装にはeBPFを活用し,Linux上で動作するアプリケーションとして開発を行った.
実装の際には予備実験にて特定したボトルネックの高速化を実施した.
そして,ランサムウェアの機能を模した簡易なプログラムを用いて,Fugaのデータ保護性能と性能上のオーバーヘッドを評価した.
評価結果から,Fugaの実装がランサムウェアの暗号化スピードに対して十分なデータ退避性能をもち,
Fugaを動作させるホストマシンへの性能上の影響が軽微であることを確認した.

\section{今後の課題}
本研究では,Fugaの設計に基づき,Linux環境を想定した実装を行なった.
しかしランサムウェア攻撃の7割以上がWindows OSを標的としている \cite{trendmicro-report} ことを考慮すると,
実用性の観点からWindows環境における実装も検討する必要がある.
本稿で示したLinux環境での実装は,Linuxカーネルの実装に強く依存しており,
Windows環境への移植には大幅な改修が求められる.

Fugaの設計では,ランサムウェアによる侵害から隔離されたデータ領域であるData Shelterを想定した.
本稿で示した実装では,Data Shelterを管理者権限を持つユーザのみがアクセス可能なディレクトリとして構築したが,
これはData Shelterの唯一の実装方法ではない.
例えば,初回の書き込み以降の変更を許可しないファイルシステムを用いる方法や,
ローカルストレージではなくリモートホストにネットワーク経由で退避データを送信するといった方法が考えられる.
特にリモートホストを利用する方法では,ランサムウェアに感染したホストと退避先のホストを物理的に分離することができるため
Data Shelterの信頼性を向上させることができる.
ただし,ホスト間の通信のスループットや遅延がデータ保護性能に影響を与える可能性があり,その影響を評価する必要がある.

Fugaは,保護対象のファイルの条件を設定可能にすることで,実用性の向上が期待される.
具体的には,ファイルパスに基づいたフィルタリング,つまり特定のパスまたはディレクトリ以下に存在するファイルのみを保護対象とする方法が考えられる.
このような設定が可能になれば,スナップショットによってファイルシステムやOS全体の状態を保存しつつ,
頻繁に更新される高価値なファイルをFugaによって保護するという補完的な運用が,より効率的に実現できる.
また,Fugaでは完全な保護が難しい大容量ファイルを保護対象から除外することで,
それらの退避処理がその他のファイルの保護性能に与える影響を軽減することができる.
