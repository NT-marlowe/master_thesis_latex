\chapter{議論}
\section{ランサムウェアの多様性への対応}
\ref{subsec:ransom-behavior}節で述べたように,ランサムウェアは暗号化を中心に様々な形態で被害を引き起こす.
ランサムウェアがファイルを侵害する方法は\ref{subsec:ransom-behavior}節の内容を踏まえて以下の3つに分類することができる.
\begin{enumerate}
  \item システムが標準的に提供する暗号化機能を利用する
  \item 独自の暗号化実装を使用する
  \item 暗号化を行わず,ファイルの上書きや削除を行う \footnote{厳密には暗号化ランサムウェアではないが,便宜上含める.}
\end{enumerate}
\ref{chap:design}章および\ref{chap:implementation}章では提案手法が(1)のタイプのランサムウェアに対してデータ保護を実現する方法を論じたが,本節では
(2)および(3)のタイプのランサムウェアに対しても提案手法を適用するための設計を検討する.

\subsubsection{独自の暗号化処理を実装するタイプ}
ランサムウェアのサンプルが入手できるという前提をおく.
サンプルを動的解析することで暗号化関数を特定し,
前節と同様にCapturerを設計する.
なお,独自の暗号化処理を実装するランサムウェアは,\ref{subsec:encrypt-algo}節で述べたように近年稀である.


\subsubsection{暗号化以外の侵害を行うタイプ}
% 暗号化を解除する対価として身代金は要求するが,ファイルを暗号化しているように見せかけて実際にはランダムなデータで上書きするようなケースを考える.
このケースではCapturerは役割を持たず,Collectorが収集するファイルメタデータのみによってファイルを退避させる必要がある.
一例として.Collectorが取得したファイルパスをEvacuation Moduleに送信し,
Evacuation Moduleが該当ファイルをData Shelterにコピーする設計が考えられる.
しかしこの設計では,Data Shelterへの書き込みの負荷が非常に高くなることが懸念される.


\section{提案手法のカバレッジ分析}
提案手法は,ファイル以外を侵害するランサムウェアには対応できない.
例としてはOSを含むディスク全体を暗号化するMamba \cite{mamba-petya} や,Windowsシステムのマスターブートレコードとファイルシステムを暗号化するPetya \cite{mamba-petya} などが挙げられる.
このタイプのランサムウェアには,ファイルシステムまたはOSイメージのスナップショットの取得による対策が必要である.

提案手法はファイルの侵害からの復旧を実現するシステムであり,二重脅迫やノーウェアランサム \cite{nowhere-ransom} におけるデータ窃取は提案手法のスコープ外である.
しかしランサムウェアのインシデントにおいてデータ窃取が行われる割合は増加傾向にある \cite{sophos-report:online}.
したがって,NDRなどの技術を活用したデータ窃取対策を検討する必要がある.

評価の結果が示すように,侵害されるファイルのサイズが大きくなると,提案手法はファイルを完全には保護できないという問題がある.
しかし一般的なエンドユーザ向けコンピュータ上のファイルのサイズ分布 \cite{file-size-dist} によると,32KB以下のファイルが全ファイル数の60\%以上を占め,
100MB以上のファイルは全ファイル数の1\%未満しか存在しない.
よって,提案手法によって保護できないような大容量のファイルが存在する場合でも,
それによる被害はシステム全体で見れば限定的であると考えられる.


\section{既存手法と提案手法の比較}
\ref{sec:ransomware-recovery}節で説明したランサムウェア被害からの復旧手法を本研究の提案手法と比較する.
\subsection{スナップショットを利用した復旧}
提案手法は\textbf{高頻度の}スナップショット取得によるデータ復旧の課題を解決する手法であり,
スナップショットを利用したデータバックアップを完全に置換するものではない.
むしろ,スナップショットと提案手法は補完的に組み合わせることができる.
動画などのサイズが大きいファイルはスナップショットによるバックアップを用いて保護し,
サイズが小さく頻繁に更新されるファイルは提案手法によってデータ更新を捕捉することで
効果的なデータ保護が期待できる.

\subsection{暗号化鍵の取得による復旧}
PayBreak \cite{kolodenker2017paybreak}は,ランサムウェアがデータの暗号化に使用する暗号化パラメータをキャプチャし,後に復号に使用する手法である.
この手法は暗号化されたデータのサイズに依存せず復旧を実現できる点で,提案手法よりも優れている側面がある.
しかし,PayBreakは原理的に共通鍵暗号方式を採用するランサムウェアに対してのみ有効である.
一方提案手法は,暗号化方式に依存せず平文データ自体を保護するため,より汎用性が高いといえる.
また,ランサムノートを提示した後に一定時間ごとに暗号化済みファイルを削除することで身代金の支払いを促すランサムウェア (e.g. Jigsaw \cite{byrne2017jigsaw})
に対しては,提案手法の方がデータ損失のリスクを効果的に抑えられる.

\subsection{SSDの特性を利用した復旧}
このタイプの手法 \cite{huang2017flashguard,baek2018ssd} にはいくつかの課題が指摘されている \cite{wang2024ransom}.
まず,SSDのファームウェアを拡張する必要があるため,ランサムウェアの急速な進化に対応するためにはファームウェアの頻繁な更新が求められる.
さらに,これらの手法は特殊なハードウェアまたは実験的なハードウェアに依存しており大規模な展開が困難である.
一方で,提案手法はソフトウェアを用いたアプローチであるため,新種のランサムウェアへの対応やシステムへの導入は比較的容易であるといえる.

\subsection{ファイルシステムの拡張による復旧}
ShiledFS \cite{shieldFS} のようにローカルファイルシステムにデータ復旧機能を追加する手法は,
正常なアプリケーションによる暗号化の誤検知や攻撃者による検知回避といった課題への対応が不十分であると指摘されている \cite{han2020effectiveness,css2024-enomoto}.
そのため,誤検知による誤った復旧や攻撃見逃しによるデータ損失のリスクが懸念される.
これに対し,提案手法は保守的な検知によって見逃しを削減することでデータ損失のリスクを軽減する.
また,誤検知が発生した場合でも無駄なバックアップが作成されるだけでデータ損失は発生しない.
なお,ローカルファイルシステムのクラウドバックアップを保護する手法 \cite{matos2018rockfs} は,提案手法とは
保護する対象が異なるアプローチであるため,本稿では比較を行わない.
