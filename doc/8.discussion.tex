\chapter{議論}
\section{提案手法のカバレッジ分析}
提案手法はファイル以外を侵害するランサムウェアには対応できない.
例としてはOSを含むディスク全体を暗号化するMamba \cite{mamba-petya} や,Windowsシステムのマスターブートレコードとファイルシステムを暗号化するPetya \cite{mamba-petya} などが挙げられる.
このタイプのランサムウェアには,ファイルシステムまたはOSイメージのスナップショットの取得による対策が必要である.

提案手法はファイルの侵害からの復旧を実現するシステムであり,二重脅迫やノーウェアランサム \cite{nowhere-ransom} におけるデータ窃取は提案手法のスコープ外である.
しかしランサムウェアのインシデントにおいてデータ窃取が行われる割合は増加傾向にある \cite{sophos-report:online}.
したがって,NDRなどの技術を活用したデータ窃取対策を検討する必要がある.

評価の結果が示すように,提案手法は侵害されるファイルのサイズが大きくなると,ファイルを完全には保護できないという問題がある.
しかしコンピュータ上のファイルのサイズ分布 \cite{file-size-dist} によると,32KB以下のファイルが全体の60\%以上を占め,
100MB以上のファイルは全体の1\%未満しか存在しない.
よって,提案手法によって保護できないような大容量のファイルが存在する場合でも,
それによる被害はシステム全体で見れば限定的であると考えられる.


\section{既存手法と提案手法の比較}
\ref{sec:ransomware-recovery}節で説明したランサムウェア被害からの復旧手法を本研究の提案手法と比較する.
\subsection{スナップショットを利用した復旧}
提案手法は\textbf{高頻度の}スナップショット取得によるデータ復旧の課題を解決する手法であり,
スナップショットを利用したデータバックアップを完全に置換するものではない.
むしろ,スナップショットと提案手法は補完的に組み合わせることができる.
動画などのサイズが大きいファイルはスナップショットによるバックアップに頼り,
サイズが小さく頻繁に更新されるファイルは提案手法によってデータ更新を補足することで
効果的なデータ保護が期待できる.

\subsection{暗号化鍵の取得による復旧}
PayBreak \cite{kolodenker2017paybreak}はランサムウェアがデータの暗号化に使用する暗号化パラメータをキャプチャし
後の復号に使用する手法であり,暗号化されるデータのサイズに関係なくデータの復旧を実現する.
この点でPayBreakは提案手法よりも優れている.
しかしPayBreakは原理的に共通鍵暗号方式を採用するランサムウェアに対してのみ有効である一方,
暗号化方式に依存せず平文データ自体を保護する提案手法は汎用性が高いといえる.
また,ランサムノートを提示した後に一定時間毎に暗号化済みファイルを削除することで身代金の支払いを促すランサムウェア (e.g. Jigsaw \cite{byrne2017jigsaw})
に対しては,提案手法の方がデータ損失のリスクを抑えられる.

\subsection{SSDの特性を利用した復旧}
このタイプの手法 \cite{huang2017flashguard,baek2018ssd} は,
SSDのファームウェアを拡張する必要があるためランサムウェアの急速な進化に対応するためにはファームウェアの頻繁な更新が必要となることと,
特殊なハードウェアまたは実験的なハードウェアに依存しており大規模な展開が困難であることが課題として指摘されている \cite{wang2024ransom}.
提案手法はソフトウェアを用いた手法であるため,新種のランサムウェアへの対応やシステムへの導入は
比較的容易であるといえる.

\subsection{ファイルシステムの拡張による復旧}
ShiledFS \cite{shieldFS} のようにローカルファイルシステムにデータ復旧機能を追加する手法は,
正常なアプリケーションによる暗号化の誤検知や攻撃者による検知回避への対応が不十分であることが指摘されている \cite{han2020effectiveness,css2024-enomoto}.
したがって誤検知の際には誤った復旧や見逃しによるデータ損失が懸念される.
提案手法は保守的な検知によって見逃しを削減するまた,提案手法において誤検知が発生した場合でも無駄なバックアップが作成されるだけでデータ損失は発生しない.
なお,ローカルファイルシステムのクラウドバックアップを保護する手法 \cite{matos2018rockfs} は提案手法とは
保護する対象が異なるアプローチであるため,ここでは比較を行わない.
