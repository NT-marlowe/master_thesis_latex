\chapter{議論}
\section{既存手法と提案手法の比較}
\ref{sec:ransomware-recovery}節で説明したランサムウェア被害からの復旧手法を本研究の提案手法と比較する.
\subsection{スナップショットを利用した復旧}
提案手法は\textbf{高頻度の}スナップショット取得によるデータ復旧の課題を解決する手法であり,
スナップショットを利用したデータバックアップを完全に置換するものではない.
むしろ,スナップショットと提案手法は補完的に組み合わせることができる.
動画などのサイズが大きいファイルはスナップショットによるバックアップに頼り,
サイズが小さく頻繁に更新されるファイルは提案手法によってデータ更新を補足することで
効果的なデータ保護が期待できる.

\subsection{暗号化鍵の取得による復旧}
PayBreak \cite{kolodenker2017paybreak}はランサムウェアがデータの暗号化に使用する暗号化パラメータをキャプチャし
後の復号に使用する手法であり,暗号化されるデータのサイズに関係なくデータの復旧を実現する.
この点でPayBreakは提案手法よりも優れている.
しかしPayBreakは原理的に共通鍵暗号方式を採用するランサムウェアに対してのみ有効である一方,
暗号化方式に依存せず平文データ自体を保護する提案手法は汎用性が高いといえる.
また,ランサムノートを提示した後に一定時間毎に暗号化済みファイルを削除することで身代金の支払いを促すランサムウェア (e.g. Jigsaw \cite{byrne2017jigsaw})
に対しては,提案手法の方がデータ損失のリスクを抑えられる.
