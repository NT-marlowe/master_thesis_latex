Ransomware threats have been increasing year by year.
In recent years, high-value organizations such as government agencies have frequently become targets of ransomware attacks, and the attack methods are becoming more systematic and sophisticated.
To counter such threats, rapid detection of ransomware and prevention of damage escalation are essential.
However, it is difficult to prevent all damages solely through detection, and there is a need for methods to enable recovery from damage after ransomware attacks.

Snapshot-based data backup methods, widely used for recovery, have limitations.
They cannot preserve data updates that occur between the latest snapshot and the onset of the attack.
Moreover, since snapshots typically target entire file systems or disks, they result in increased storage consumption.

In this study, we propose a novel method to protect data from ransomware attacks.
The proposed method reactively backs up data on a per-file basis and protects it in an isolated area, preserving the data in the state it was in just before the ransomware attack.
Specifically, the method identifies suspicious processes using an online detection approach and hooks file-compromising operations executed by those processes.
When a hooked operation is triggered, the data of the targeted file is copied to an isolated area, protecting it from ransomware.
This approach prevents data loss due to updates and enables backing up only the necessary files.
% 
In our study, we implemented the proposed method in a Linux environment, assuming ransomware using a standard encryption library.
Using a ransomware-mimicking program, we conducted experiments to evaluate data protection performance and overhead for various file sizes.
The results demonstrated that the proposed method provides sufficient data protection performance against ransomware encryption speed, with minimal performance overhead.
