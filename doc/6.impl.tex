\chapter{Fugaの実装}
\ref{chap:design}章で述べたFugaの設計をもとにFugaを実装した.
本稿の実装は,TMP節で述べたケースの実例として,OpenSSLを利用して暗号化を行うランサムウェアを対象としている.
カーネル空間のプログラムはeBPF \cite{WhatiseB81:online} を利用して,
ユーザ空間のプログラムはGo言語を用いて実装した.
.本稿の執筆時点においてeBPFは実用上Linux環境のみで利用可能であるため,
Linux環境にて実装を行った.
実装に使用したシステムの環境を\tabref{tab:impl-env}に示す.
\begin{table}[tb]
  \caption{Implementation environment for Fuga.}
  \label{tab:impl-env}
  \hbox to\hsize{\hfil
    \begin{tabular}{l|lll}
      \hline
      Linuxカーネル  & 6.3.0-060300-generic \\
      CPUアーキテクチャ & amd64                \\
      clang      & 14.0.0-1ubuntu1.1    \\
      Goコンパイラ    & go1.22.5 linux/amd64 \\
    \end{tabular}\hfil}
\end{table}
