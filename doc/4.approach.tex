\chapter{提案手法}
本章では,ランサムウェアによるファイル侵害からファイルデータを保護するための新しい手法を提案する.
ここでいう「ファイルの侵害」とは暗号化・上書き・削除などの操作によってファイルを利用不可能にすることを指す.
提案手法はランサムウェアのオンライン検知手法を使用し,ランサムウェアの疑いがあるユーザプロセスを特定する.
そして特定されたプロセスが実行する,事前に定義したファイル侵害の処理をフックする.
フックされた処理が実行されると,その処理のエントリーポイントにおいて侵害対象のファイルはランサムウェアから隔離された領域にコピーされる.
(ランサムウェアが十分長い時間活動して検知可能になったらそのプロセスはkillしておk.提案手法は検知まで耐えてくれれば保護としては十分である.)
ランサムウェアによって侵害される直前の状態でファイルのデータが保護されるため,ランサムウェア被害からの復旧が可能となる.


\section{既存の課題との対応}
\ref{sec:recovery-challenges}節で述べた課題を提案手法が緩和または解決する方法を示す.
\subsection{スナップショットによる復旧の課題}
提案手法はランサムウェアによって侵害されるファイルを,侵害の直前にバックアップする.


