\chapter{提案手法}
本章では,ランサムウェアによるファイル侵害からファイルデータを保護するための新しい手法を提案する.
ここでいう「ファイルの侵害」とは暗号化・上書き・削除などの操作によってファイルを利用不可能にすることを指す.
提案手法はランサムウェアのオンライン検知手法を使用し,ランサムウェアの疑いがあるユーザプロセスを特定する.
検知手法は閾値を調整するなどしてランサムウェアの見逃しを削減するように設定する.
そして特定されたプロセスが実行する,事前に定義したファイル侵害の処理をフックする.
フックされた処理が実行されると,その処理のエントリーポイントにおいて侵害対象のファイルはランサムウェアから隔離された領域にコピーされる.
% (ランサムウェアが十分長い時間活動して検知可能になったらそのプロセスはkillしておk.提案手法は検知まで耐えてくれれば保護としては十分である.)
ランサムウェアによって侵害される直前の状態でファイルのデータが保護されるため,ランサムウェア被害からの復旧が可能となる.

\section{既存手法の課題への対応}
\ref{sec:recovery-challenges}節で述べた課題を提案手法が緩和または解決する方法を示す.
\subsection{スナップショットによる復旧の課題への対応}
提案手法は,理想的にはランサムウェアによって侵害されるデータのみをバックアップし,
正常なアプリケーションによって更新されたデータはバックアップしない.
これによりバックアップデータによるストレージ容量の圧迫を軽減する.
さらに,スナップショット方式ではどの時点のスナップショットを利用して復旧を行うかを選択する必要があるが,
提案手法ではランサムウェアによる侵害の直前のデータが保護される.
つまり提案手法を用いると復旧時に使用すべきバックアップデータは単一かつ最新に保たれるため,データ復旧の手順の簡略化と作業コストの削減が期待できる.

提案手法はスナップショット方式において発生するデータ損失を回避することができる.
第一に,リアクティブなバックアップ取得によって,最新のバックアップ以降に発生したデータ更新は保護される.
第二に,提案手法はバックアップが必要なデータのみを保護するため,
正常なアプリケーションによる高頻度のデータ更新が発生する環境におけるデータ損失は発生しない.

実際には誤検知と見逃しの両方を完全に排除することは現実的ではなく,
提案手法が採用するランサムウェアのオンライン検知手法が誤検知を行った場合に
余計なデータバックアップが作成される.
しかし依然として,提案手法はバックアップによるストレージ容量の消費を軽減し,上述したデータ損失を防ぐことができる.

\subsection{検知の課題への対応}
提案手法は見逃し率を小さく抑えるように調整された検知手法を利用し,各ユーザプロセスがランサムウェアの疑いがあるかどうかを判別する.
この調整によって,実際にはランサムウェアであるプロセスを検出できずデータが侵害される可能性を減らす.


