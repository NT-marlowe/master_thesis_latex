\chapter{提案手法}
\label{chap:approach}
本章では,ランサムウェアによるファイル侵害からファイルデータを保護する新しい手法である\textbf{Fuga}\footnote{ふーが.フーガ.}を提案する.
ここでいう「ファイルの侵害」とは暗号化,上書き,削除などの操作によってファイルを利用不可能にすることを指す.
Fugaは,ランサムウェアのファイル侵害に対してリアクティブに,侵害されるファイル単位でバックアップを取得し,
侵害直前の状態でデータを保護する.これにより被害からの速やかな復旧を可能にする.

Fugaはランサムウェアのオンライン検知手法を利用し,ランサムウェアの疑いがあるユーザプロセスを特定する.
検知手法は閾値を調整するなどしてランサムウェアの見逃しを削減するように設定する.
そして特定されたプロセスが実行する,事前定義されたファイル侵害の処理をフックする.
フックされた処理が実行されると,その処理のエントリーポイントにおいて侵害対象のファイルはランサムウェアから隔離された領域にコピーされる.
% (ランサムウェアが十分長い時間活動して検知可能になったらそのプロセスはkillしておk.Fugaは検知まで耐えてくれれば保護としては十分である.)
これによりランサムウェアによって侵害される直前の状態でファイルのデータが保護されるため,ランサムウェア被害からの復旧が可能となる.

\section{既存手法の課題への対応}
\ref{sec:recovery-challenges}節で述べた課題をFugaが緩和または解決する方法を示す.
\subsection{スナップショットによる復旧の課題への対応}
Fugaは,理想的にはランサムウェアによって侵害されるデータのみをバックアップし,
正常なアプリケーションによって更新されたデータはバックアップしない.
これによりバックアップデータによるストレージ容量の圧迫を軽減することができる.
さらに,複数のスナップショットを取得してバックアップを行う場合,重複して保存されているデータがストレージ容量を消費するが,
Fugaにおいては重複データは発生しない.
% さらに,スナップショット方式ではどの時点のスナップショットを利用して復旧を行うかを選択する必要があるが,
これは,Fugaを用いるとランサムウェアによる侵害の直前のデータが保護され,バックアップデータは単一かつ最新に保たれるからである.
% つまりFugaを用いると復旧時に使用すべきバックアップデータは単一かつ最新に保たれるため,データ復旧の手順の簡略化と作業コストの削減が期待できる.

Fugaはスナップショット方式において問題となるデータ損失を回避することができる.
第一に,Fugaではファイル侵害の直前にバックアップを取得するため,
% スナップショット方式において問題となる,
最新のバックアップ取得後から攻撃発生までの間という時間的なギャップが存在しない.
したがってスナップショット方式にて発生する,この時間的なギャップに起因するデータ損失を回避することができる.
% リアクティブなバックアップ取得によって,最新のバックアップ以降に発生したデータ更新は保護される.
第二に,Fugaはバックアップが必要なデータのみを保護する仕組みであるため,
ランサムウェアによって侵害された後のデータと正常なアプリケーションによって更新されたデータが
バックアップに混在することがない.
% 正常なアプリケーションによる高頻度のデータ更新が発生する環境であってもデータ損失が発生しない.

実際には誤検知と見逃しの両方を完全に排除することは現実的ではなく,
Fugaが採用するランサムウェアのオンライン検知手法が誤検知を行った場合に
無駄なデータバックアップが作成される.
しかし依然として,Fugaはバックアップによるストレージ容量の消費を軽減し,上述したデータ損失を防ぐことができる.

\subsection{検知の課題への対応}
Fugaでは,見逃し率を抑えてデータ損失を防ぐ「保守的な検知」ポリシーを採用する.
% 見逃し率を小さくしてデータ損失を防ぐ検知手法を「保守的な検知」と呼ぶことにする.
保守的な検知では,ユーザプロセスがランサムウェアである可能性が低くても,侵害のリスクを回避するために早期に検知を行う.
これにより,実際にランサムウェアであるプロセスを見逃す可能性を減らし,検知レイテンシも短縮される.
検知レイテンシの短縮は,プロセスの「ランサムウェアらしさ」(たとえばRedemption \cite{kharraz2017redemption}において計算される悪意スコア)
が十分に確からしくない段階でも,「ランサムウェアである」と判断するためである.
% Fugaは保守的な検知手法を利用し,各ユーザプロセスがランサムウェアの疑いがあるかどうかを判別する.
% この調整によって,実際にはランサムウェアであるプロセスを検出できずデータが侵害される可能性を減らし,検知レイテンシを短縮する.
% 検知レイテンシが短縮される理由は,ユーザプロセスの「ランサムウェアらしさ」が十分確からしくなっていない,例えばRedemption \cite{kharraz2017redemption}
% において計算される悪意スコアが高くない,時点で「ランサムウェアである」と判断することができるからである.

保守的な検知の設定により誤検知は増加するが,Fugaでは誤検知による影響を最小限に抑えている.
% 見逃しを減らす検知ポリシーを設定するとトレードオフとして誤検知が増加する.
具体的には,誤検知が発生しても余計にバックアップされるファイルが増えるだけであり,ストレージ容量への影響は小さい.
また,データのバックアップを取得している間に,フック対象のプロセスの「ランサムウェアらしさ」がさらに明確になることが期待される.
「ランサムウェアらしさ」が高くなった場合には該当プロセスを停止すればよく,低くなった場合にはバックアップ対象から除外することで,効率的なデータ保護が可能となる.
% しかしFugaでは,誤検知が発生したとしても一部のファイルが余計にバックアップされるだけであり,
% ストレージ容量に発生する影響は小さいといえる.
% Fugaを用いてデータバックアップを取得している間,一定時間の経過後にフック対象のプロセスの「ランサムウェアらしさ」は
% 保守的な検知においても十分に高くなるか,または十分に低くなることが期待される.
% 前者では該当プロセスを停止すればよく,後者ではバックアップ取得の対象となるプロセス群から該当プロセスを除外すればよい.



