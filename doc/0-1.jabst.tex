% ランサムウェアは年々その脅威が高まっている.
% 政府機関などの高価値な組織が攻撃の対象となることが近年多く,攻撃手法も体系化,高度化する傾向にある.
% このような脅威に対抗するためにはランサムウェアの迅速な検知と被害の拡大防止が不可欠だが,
% 検知のみによってすべての被害を未然に防ぐことは難しく,ランサムウェア攻撃後に被害の復旧を実現するための手法も求められている.
% しかし,復旧のためのデータバックアップ手法として広く利用されているスナップショットは,
% 最新のスナップショット取得後から攻撃発生までの間のデータ更新を保存できず,
% またファイルシステムやディスクなどを対象として取得されるため,ストレージ消費量の増加も課題となる.

% % 本研究ではランサムウェアのファイル侵害の直前にデータを隔離領域へ退避させることでデータ復旧を実現する手法を提案する.
% 本研究では,ランサムウェアによる侵害からデータを保護する新たな手法を提案する.
% 提案手法は侵害されるファイル単位でリアクティブにデータのバックアップを取得し,
% ランサムウェアによる侵害直前の状態でデータを隔離領域に保護する.
% 提案手法ではオンライン検知手法を利用してランサムウェアの疑いがあるプロセスを特定し,そのプロセスが実行するファイル侵害操作をフックする.
% フックされた処理が実行されると,侵害対象のファイルのデータをランサムウェアから隔離された領域にコピーする.
% これによってデータ更新の損失を防ぎ,必要なファイルのみをバックアップする.
% % 
% 本研究では標準的な暗号化ライブラリを使用するランサムウェアを想定し,Linux環境において実装を行なった.
% ランサムウェアを模したプログラムによる実験では,
% 様々なファイルサイズにおけるデータ保護性能とオーバーヘッドを評価した.
% その結果,提案手法はランサムウェアの暗号化速度に対して十分なデータ保護性能を示し,
% 性能上のオーバーヘッドも軽微であることが確認された.

ランサムウェアは年々その脅威が高まっている.
政府機関などの高価値な組織が攻撃の対象となることが近年多く,攻撃手法も体系化,高度化する傾向にある.
このような脅威に対抗するためには,ランサムウェアの迅速な検知と被害の拡大防止が不可欠である.
しかし,検知のみによってすべての被害を未然に防ぐことは難しく,ランサムウェア攻撃後に被害の復旧を実現するための手法も求められている.
一方で,復旧のためのデータバックアップ手法として広く利用されているスナップショットは,
最新のスナップショット取得後から攻撃発生までの間に行われたデータ更新を保存できないという課題がある.
さらに,スナップショットはファイルシステムやディスク全体を対象として取得されるため,ストレージ消費量が増加する点も問題となる.

本研究では,ランサムウェアによる侵害からデータを保護する新たな手法を提案する.
提案手法は,侵害されるファイル単位でリアクティブにデータのバックアップを取得し,ランサムウェアによる侵害直前の状態でデータを隔離領域に保護する.
具体的には,オンライン検知手法を利用してランサムウェアの疑いがあるプロセスを特定し,そのプロセスが実行するファイル侵害操作をフックする.
フックされた処理が実行されると,侵害対象のファイルのデータをランサムウェアから隔離された領域にコピーする.
これにより,データ更新の損失を防ぎつつ,必要なファイルのみをバックアップすることが可能となる.
% 
本研究では,標準的な暗号化ライブラリを使用するランサムウェアを想定し,Linux環境において提案手法を実装した.
ランサムウェアを模したプログラムを用いた実験では,様々なファイルサイズにおけるデータ保護性能とオーバーヘッドを評価した.
その結果,提案手法はランサムウェアの暗号化速度に対して十分なデータ保護性能を示し,性能上のオーバーヘッドも軽微であることが確認された.
