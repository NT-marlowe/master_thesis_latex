ランサムウェアの脅威に対応するためには,ランサムウェアの活動を迅速に検知して被害の拡大を防ぐだけでなく,
仮に被害を受けた場合でもデータを復旧することができる手法が必要となる.
定期的なスナップショットによる復旧は一般的な手法だが,バックアップの粒度の粗さから
頻繁な取得は難しく,その結果データ損失が発生するおそれがある.
本研究ではランサムウェアのファイル侵害の直前にデータを隔離領域へ退避させることでデータ復旧を実現する
システムを提案する.
提案手法はファイル単位のバックアップをリアクティブに取得することでスナップショット方式の課題を克服し,誤検知時のコストも軽減する.
本稿では提案手法の設計を行い,ランサムウェアのファイル侵害方法に応じた実装方針を示した.
さらにeBPFを用いた実装を行い,
様々なファイルサイズにおけるデータ保護性能とオーバーヘッドを評価した.
これにより,ランサムウェアからデータを保護するシステムとして提案手法が実用的であることを示した.
