\chapter{序論}
\section{背景}
ランサムウェアは年々その脅威を増しており,個人や組織の活動を阻害し,社会全体に大きな経済的損失をもたらしている.
特に,ランサムウェア攻撃の対象は個人から高価値な組織へと移行しつつあり \cite{sophos-report:online,early-detection},
その影響は組織内にとどまらず社会全体に波及する可能性がある.
たとえば,Covid-19 パンデミック下の医療機関を狙った攻撃 \cite{Covid19R19:online} や,
物流インフラを標的とした攻撃 \cite{nagoya-port-attack} は,その象徴的な事例である.
また,Ransomware as a Service(RaaS)モデルの普及により,攻撃者は金銭を対価にランサムウェアを配布したり,
攻撃を実施したりすることが組織的に行われている.
さらに,Bitcoin に代表される暗号通貨の存在が,攻撃者に匿名での身代金受け取りを可能にし,ランサムウェア攻撃の増加を助長している.
こうした背景の中,2023 年にはランサムウェア攻撃者に支払われた身代金の総額が11億ドルに達したと報告されており \cite{Ransomwa86:online},
SOPHOS 社の調査 \cite{sophos-report:online} によれば,2024 年には 59\% の組織がランサムウェア攻撃を経験している.

ランサムウェアの脅威に対応するためには,その存在を迅速に検知して被害の拡大を防ぐことが重要である.
しかし,被害状況の拡大を鑑みると,検知のみによって全ての被害を未然に防ぐことは難しいといえる.
したがって,ランサムウェアの活動を未然に防げなかった場合でも,攻撃者に身代金を支払わずに被害からの復旧を実現する手法が必要となる.
復旧の手法としてシステムの定期的なスナップショットの取得が存在するが,
スナップショット方式ではバックアップ取得からランサムウェア被害までの間に発生するデータ更新
が失われるリスクがある \cite{wang2024ransom}.
これは,スナップショットがディスクやファイルシステムの単位で取得されるので,ディスク I/O スループットの限界やストレージ消費量の問題により頻繁な取得が難しいためである
\cite{wang2024ransom, veena2021incremental}.

本研究では,ランサムウェアによるファイル侵害からデータを保護するための新たな手法であるFugaを提案する.
Fugaはランサムウェアによる侵害の直前に,ファイル単位でリアクティブにバックアップを取得する.
これにより,各ファイルのデータがランサムウェアによる侵害直前の状態で保存され,
バックアップ取得から攻撃までの間に発生するデータ更新の損失を防ぐことができる.
% スナップショット方式で発生しうるデータ損失を回避する.
% バックアップ取得からランサムウェア被害までの間に発生するデータ更新を保護する.
さらに,バックアップの単位がファイルであるため,誤検知によって無駄なデータ退避が発生した場合でもストレージ消費量は比較的小さい.
% 最新のスナップショット取得からランサムウェア被害発生までのデータ更新も保護する
\section{本研究の貢献}
本研究の貢献を以下に示す.
\begin{enumerate}
  \item ランサムウェアによるファイル侵害に対し,リアクティブにファイル単位でデータを保護するシステムFugaを提案した.

        % \item 提案手法のアプローチを実現するための設計を示した.ファイル侵害の方法の観点からランサムウェアの分類を行い,各分類ごとの実装の概要を整理した.

        % \item 標準的な暗号化ライブラリに依存して暗号化を行うランサムウェアを想定した実装を示し,
        % eBPFを用いてFugaを実装した.
  \item 標準的な暗号化ライブラリを使用するランサムウェアを想定し,eBPFを用いたFugaの実装を行なった.
        % eBPFを用いてFugaを実装した.

        % \item ファイルを保護する性能とパフォーマンスに関する評価を実施した.保護性能の評価として,様々なサイズのファイルを暗号化した際に退避できるデータの割合を計測した.パフォーマンス評価としてCPU使用率およびディスクI/O使用率を計測し,Fugaがデータを退避するスループットを推定した.
  \item Fugaのファイルデータ保護性能とパフォーマンスを評価した.保護性能では,暗号化対象ファイルにおける退避に成功したデータの割合を,
        パフォーマンスではFugaのオーバーヘッドとデータ退避のスループットを計測し,ランサムウェアの暗号化速度に対して十分なスループットを確認した.
\end{enumerate}

\section{構成}
本論文は以下の通り構成される.
第\ref{chap:ransomware-overview}章では本研究の対象としているランサムウェアを概観し,
ランサムウェアの歴史から特徴に基づく分類,近年の傾向を紹介する.
第\ref{chap:ransomware-defense}章にて既存のランサムウェア対策手法を整理し,
その上でランサムウェア被害からの復旧を実現する既存手法の課題を考察する.
それを踏まえ,第\ref{chap:approach}章では課題に対するアプローチとしてFugaを提案し,
第\ref{chap:design}章にてFugaの設計を示す.
第\ref{chap:implementation}章ではFugaの実装について述べる.
第\ref{chap:eval}章ではFugaの評価を論じる.
第\ref{chap:discussion}にてFugaの拡張性や限界を議論し,既存手法との比較を行う.
第\ref{chap:conclusion}で本研究をまとめ,今後の課題を示す.
