\chapter{序論}
% \section{はじめに}
ランサムウェアの被害は年々拡大しており,個人および組織の活動を阻害し大きな経済的損失をもたらす脅威となっている.
2023年にランサムウェア攻撃者に支払われた身代金の総額は11億ドルに達したと報告されており\cite{Ransomwa86:online},
SOPHOS社の調査 \cite{sophos-report:online} によると,2024年には59\%もの組織がランサムウェア攻撃を受けた.
また,医療機関や政府組織などの重要な組織を狙ったランサムウェア攻撃が近年増加しており \cite{sophos-report:online},
ランサムウェアは社会的に深刻な問題といえる.

ランサムウェアの存在を早期に特定して対処する検知手法の開発と導入は不可欠だが,
被害状況の拡大を鑑みるに,検知によって全ての被害を未然に防ぐことは難しいといえる.
したがって,ランサムウェアの活動を未然に防げなかった場合でも,攻撃者に身代金を支払わずに被害からの復旧を実現する手法が必要となる.
復旧の手法としてシステムの定期的なスナップショットの取得が存在するが,
スナップショット方式ではバックアップ取得からランサムウェア被害までの間に発生するデータ更新
が失われるリスクがある \cite{wang2024ransom}.
これは,スナップショットがディスクやファイルシステムの単位で取得されるので,ディスク I/O スループットの限界やストレージ消費量の問題により頻繁な取得が難しいためである \cite{veena2021incremental, wang2024ransom}.

本研究では,ランサムウェアによるファイル侵害からデータを保護するための新たな手法を提案する.
提案手法はランサムウェアによる侵害の直前に,ファイル単位でリアクティブにバックアップを取得する.
これにより,各ファイルのデータがランサムウェアによる侵害直前の状態で保存され,
バックアップ取得から攻撃までの間に発生するデータ更新の損失を防ぐことができる.
% スナップショット方式で発生しうるデータ損失を回避する.
% バックアップ取得からランサムウェア被害までの間に発生するデータ更新を保護する.
さらに,バックアップの単位がファイルであるため,誤検知によって無駄なデータ退避が発生した場合でもストレージ消費量は比較的小さい.
% 最新のスナップショット取得からランサムウェア被害発生までのデータ更新も保護する
本研究の貢献を以下に示す.
\begin{enumerate}
  \item ランサムウェアによるファイル侵害に対し,リアクティブにファイル単位でデータを保護するシステムを提案した.

  \item 提案手法のアプローチを実現するための設計を示した.ファイル侵害の方法の観点からランサムウェアの分類を行い,各分類ごとの実装の概要を整理した.

        % \item 標準的な暗号化ライブラリに依存して暗号化を行うランサムウェアを想定した実装を示し,
        % eBPFを用いてFugaを実装した.
  \item 標準的な暗号化ライブラリを使用するランサムウェアを想定し,eBPFを用いた実装を行なった.
        % eBPFを用いてFugaを実装した.

        % \item ファイルを保護する性能とパフォーマンスに関する評価を実施した.保護性能の評価として,様々なサイズのファイルを暗号化した際に退避できるデータの割合を計測した.パフォーマンス評価としてCPU使用率およびディスクI/O使用率を計測し,Fugaがデータを退避するスループットを推定した.
  \item 提案手法のファイルデータ保護性能とパフォーマンスを評価した.保護性能では,暗号化対象ファイルにおける退避に成功したデータの割合を,
        パフォーマンスでは提案手法のオーバーヘッドとデータ退避のスループットを計測し,ランサムウェアの暗号化速度に対して十分なスループットを確認した.
\end{enumerate}
